
% Default to the notebook output style

    


% Inherit from the specified cell style.




    
\documentclass[11pt]{article}

    
    
    \usepackage[T1]{fontenc}
    % Nicer default font (+ math font) than Computer Modern for most use cases
    \usepackage{mathpazo}

    % Basic figure setup, for now with no caption control since it's done
    % automatically by Pandoc (which extracts ![](path) syntax from Markdown).
    \usepackage{graphicx}
    % We will generate all images so they have a width \maxwidth. This means
    % that they will get their normal width if they fit onto the page, but
    % are scaled down if they would overflow the margins.
    \makeatletter
    \def\maxwidth{\ifdim\Gin@nat@width>\linewidth\linewidth
    \else\Gin@nat@width\fi}
    \makeatother
    \let\Oldincludegraphics\includegraphics
    % Set max figure width to be 80% of text width, for now hardcoded.
    \renewcommand{\includegraphics}[1]{\Oldincludegraphics[width=.8\maxwidth]{#1}}
    % Ensure that by default, figures have no caption (until we provide a
    % proper Figure object with a Caption API and a way to capture that
    % in the conversion process - todo).
    \usepackage{caption}
    \DeclareCaptionLabelFormat{nolabel}{}
    \captionsetup{labelformat=nolabel}

    \usepackage{adjustbox} % Used to constrain images to a maximum size 
    \usepackage{xcolor} % Allow colors to be defined
    \usepackage{enumerate} % Needed for markdown enumerations to work
    \usepackage{geometry} % Used to adjust the document margins
    \usepackage{amsmath} % Equations
    \usepackage{amssymb} % Equations
    \usepackage{textcomp} % defines textquotesingle
    % Hack from http://tex.stackexchange.com/a/47451/13684:
    \AtBeginDocument{%
        \def\PYZsq{\textquotesingle}% Upright quotes in Pygmentized code
    }
    \usepackage{upquote} % Upright quotes for verbatim code
    \usepackage{eurosym} % defines \euro
    \usepackage[mathletters]{ucs} % Extended unicode (utf-8) support
    \usepackage[utf8x]{inputenc} % Allow utf-8 characters in the tex document
    \usepackage{fancyvrb} % verbatim replacement that allows latex
    \usepackage{grffile} % extends the file name processing of package graphics 
                         % to support a larger range 
    % The hyperref package gives us a pdf with properly built
    % internal navigation ('pdf bookmarks' for the table of contents,
    % internal cross-reference links, web links for URLs, etc.)
    \usepackage{hyperref}
    \usepackage{longtable} % longtable support required by pandoc >1.10
    \usepackage{booktabs}  % table support for pandoc > 1.12.2
    \usepackage[inline]{enumitem} % IRkernel/repr support (it uses the enumerate* environment)
    \usepackage[normalem]{ulem} % ulem is needed to support strikethroughs (\sout)
                                % normalem makes italics be italics, not underlines
    

    
    
    % Colors for the hyperref package
    \definecolor{urlcolor}{rgb}{0,.145,.698}
    \definecolor{linkcolor}{rgb}{.71,0.21,0.01}
    \definecolor{citecolor}{rgb}{.12,.54,.11}

    % ANSI colors
    \definecolor{ansi-black}{HTML}{3E424D}
    \definecolor{ansi-black-intense}{HTML}{282C36}
    \definecolor{ansi-red}{HTML}{E75C58}
    \definecolor{ansi-red-intense}{HTML}{B22B31}
    \definecolor{ansi-green}{HTML}{00A250}
    \definecolor{ansi-green-intense}{HTML}{007427}
    \definecolor{ansi-yellow}{HTML}{DDB62B}
    \definecolor{ansi-yellow-intense}{HTML}{B27D12}
    \definecolor{ansi-blue}{HTML}{208FFB}
    \definecolor{ansi-blue-intense}{HTML}{0065CA}
    \definecolor{ansi-magenta}{HTML}{D160C4}
    \definecolor{ansi-magenta-intense}{HTML}{A03196}
    \definecolor{ansi-cyan}{HTML}{60C6C8}
    \definecolor{ansi-cyan-intense}{HTML}{258F8F}
    \definecolor{ansi-white}{HTML}{C5C1B4}
    \definecolor{ansi-white-intense}{HTML}{A1A6B2}

    % commands and environments needed by pandoc snippets
    % extracted from the output of `pandoc -s`
    \providecommand{\tightlist}{%
      \setlength{\itemsep}{0pt}\setlength{\parskip}{0pt}}
    \DefineVerbatimEnvironment{Highlighting}{Verbatim}{commandchars=\\\{\}}
    % Add ',fontsize=\small' for more characters per line
    \newenvironment{Shaded}{}{}
    \newcommand{\KeywordTok}[1]{\textcolor[rgb]{0.00,0.44,0.13}{\textbf{{#1}}}}
    \newcommand{\DataTypeTok}[1]{\textcolor[rgb]{0.56,0.13,0.00}{{#1}}}
    \newcommand{\DecValTok}[1]{\textcolor[rgb]{0.25,0.63,0.44}{{#1}}}
    \newcommand{\BaseNTok}[1]{\textcolor[rgb]{0.25,0.63,0.44}{{#1}}}
    \newcommand{\FloatTok}[1]{\textcolor[rgb]{0.25,0.63,0.44}{{#1}}}
    \newcommand{\CharTok}[1]{\textcolor[rgb]{0.25,0.44,0.63}{{#1}}}
    \newcommand{\StringTok}[1]{\textcolor[rgb]{0.25,0.44,0.63}{{#1}}}
    \newcommand{\CommentTok}[1]{\textcolor[rgb]{0.38,0.63,0.69}{\textit{{#1}}}}
    \newcommand{\OtherTok}[1]{\textcolor[rgb]{0.00,0.44,0.13}{{#1}}}
    \newcommand{\AlertTok}[1]{\textcolor[rgb]{1.00,0.00,0.00}{\textbf{{#1}}}}
    \newcommand{\FunctionTok}[1]{\textcolor[rgb]{0.02,0.16,0.49}{{#1}}}
    \newcommand{\RegionMarkerTok}[1]{{#1}}
    \newcommand{\ErrorTok}[1]{\textcolor[rgb]{1.00,0.00,0.00}{\textbf{{#1}}}}
    \newcommand{\NormalTok}[1]{{#1}}
    
    % Additional commands for more recent versions of Pandoc
    \newcommand{\ConstantTok}[1]{\textcolor[rgb]{0.53,0.00,0.00}{{#1}}}
    \newcommand{\SpecialCharTok}[1]{\textcolor[rgb]{0.25,0.44,0.63}{{#1}}}
    \newcommand{\VerbatimStringTok}[1]{\textcolor[rgb]{0.25,0.44,0.63}{{#1}}}
    \newcommand{\SpecialStringTok}[1]{\textcolor[rgb]{0.73,0.40,0.53}{{#1}}}
    \newcommand{\ImportTok}[1]{{#1}}
    \newcommand{\DocumentationTok}[1]{\textcolor[rgb]{0.73,0.13,0.13}{\textit{{#1}}}}
    \newcommand{\AnnotationTok}[1]{\textcolor[rgb]{0.38,0.63,0.69}{\textbf{\textit{{#1}}}}}
    \newcommand{\CommentVarTok}[1]{\textcolor[rgb]{0.38,0.63,0.69}{\textbf{\textit{{#1}}}}}
    \newcommand{\VariableTok}[1]{\textcolor[rgb]{0.10,0.09,0.49}{{#1}}}
    \newcommand{\ControlFlowTok}[1]{\textcolor[rgb]{0.00,0.44,0.13}{\textbf{{#1}}}}
    \newcommand{\OperatorTok}[1]{\textcolor[rgb]{0.40,0.40,0.40}{{#1}}}
    \newcommand{\BuiltInTok}[1]{{#1}}
    \newcommand{\ExtensionTok}[1]{{#1}}
    \newcommand{\PreprocessorTok}[1]{\textcolor[rgb]{0.74,0.48,0.00}{{#1}}}
    \newcommand{\AttributeTok}[1]{\textcolor[rgb]{0.49,0.56,0.16}{{#1}}}
    \newcommand{\InformationTok}[1]{\textcolor[rgb]{0.38,0.63,0.69}{\textbf{\textit{{#1}}}}}
    \newcommand{\WarningTok}[1]{\textcolor[rgb]{0.38,0.63,0.69}{\textbf{\textit{{#1}}}}}
    
    
    % Define a nice break command that doesn't care if a line doesn't already
    % exist.
    \def\br{\hspace*{\fill} \\* }
    % Math Jax compatability definitions
    \def\gt{>}
    \def\lt{<}
    % Document parameters
    \title{Visualizing Earnings Based On College Majors}
    
    
    

    % Pygments definitions
    
\makeatletter
\def\PY@reset{\let\PY@it=\relax \let\PY@bf=\relax%
    \let\PY@ul=\relax \let\PY@tc=\relax%
    \let\PY@bc=\relax \let\PY@ff=\relax}
\def\PY@tok#1{\csname PY@tok@#1\endcsname}
\def\PY@toks#1+{\ifx\relax#1\empty\else%
    \PY@tok{#1}\expandafter\PY@toks\fi}
\def\PY@do#1{\PY@bc{\PY@tc{\PY@ul{%
    \PY@it{\PY@bf{\PY@ff{#1}}}}}}}
\def\PY#1#2{\PY@reset\PY@toks#1+\relax+\PY@do{#2}}

\expandafter\def\csname PY@tok@w\endcsname{\def\PY@tc##1{\textcolor[rgb]{0.73,0.73,0.73}{##1}}}
\expandafter\def\csname PY@tok@c\endcsname{\let\PY@it=\textit\def\PY@tc##1{\textcolor[rgb]{0.25,0.50,0.50}{##1}}}
\expandafter\def\csname PY@tok@cp\endcsname{\def\PY@tc##1{\textcolor[rgb]{0.74,0.48,0.00}{##1}}}
\expandafter\def\csname PY@tok@k\endcsname{\let\PY@bf=\textbf\def\PY@tc##1{\textcolor[rgb]{0.00,0.50,0.00}{##1}}}
\expandafter\def\csname PY@tok@kp\endcsname{\def\PY@tc##1{\textcolor[rgb]{0.00,0.50,0.00}{##1}}}
\expandafter\def\csname PY@tok@kt\endcsname{\def\PY@tc##1{\textcolor[rgb]{0.69,0.00,0.25}{##1}}}
\expandafter\def\csname PY@tok@o\endcsname{\def\PY@tc##1{\textcolor[rgb]{0.40,0.40,0.40}{##1}}}
\expandafter\def\csname PY@tok@ow\endcsname{\let\PY@bf=\textbf\def\PY@tc##1{\textcolor[rgb]{0.67,0.13,1.00}{##1}}}
\expandafter\def\csname PY@tok@nb\endcsname{\def\PY@tc##1{\textcolor[rgb]{0.00,0.50,0.00}{##1}}}
\expandafter\def\csname PY@tok@nf\endcsname{\def\PY@tc##1{\textcolor[rgb]{0.00,0.00,1.00}{##1}}}
\expandafter\def\csname PY@tok@nc\endcsname{\let\PY@bf=\textbf\def\PY@tc##1{\textcolor[rgb]{0.00,0.00,1.00}{##1}}}
\expandafter\def\csname PY@tok@nn\endcsname{\let\PY@bf=\textbf\def\PY@tc##1{\textcolor[rgb]{0.00,0.00,1.00}{##1}}}
\expandafter\def\csname PY@tok@ne\endcsname{\let\PY@bf=\textbf\def\PY@tc##1{\textcolor[rgb]{0.82,0.25,0.23}{##1}}}
\expandafter\def\csname PY@tok@nv\endcsname{\def\PY@tc##1{\textcolor[rgb]{0.10,0.09,0.49}{##1}}}
\expandafter\def\csname PY@tok@no\endcsname{\def\PY@tc##1{\textcolor[rgb]{0.53,0.00,0.00}{##1}}}
\expandafter\def\csname PY@tok@nl\endcsname{\def\PY@tc##1{\textcolor[rgb]{0.63,0.63,0.00}{##1}}}
\expandafter\def\csname PY@tok@ni\endcsname{\let\PY@bf=\textbf\def\PY@tc##1{\textcolor[rgb]{0.60,0.60,0.60}{##1}}}
\expandafter\def\csname PY@tok@na\endcsname{\def\PY@tc##1{\textcolor[rgb]{0.49,0.56,0.16}{##1}}}
\expandafter\def\csname PY@tok@nt\endcsname{\let\PY@bf=\textbf\def\PY@tc##1{\textcolor[rgb]{0.00,0.50,0.00}{##1}}}
\expandafter\def\csname PY@tok@nd\endcsname{\def\PY@tc##1{\textcolor[rgb]{0.67,0.13,1.00}{##1}}}
\expandafter\def\csname PY@tok@s\endcsname{\def\PY@tc##1{\textcolor[rgb]{0.73,0.13,0.13}{##1}}}
\expandafter\def\csname PY@tok@sd\endcsname{\let\PY@it=\textit\def\PY@tc##1{\textcolor[rgb]{0.73,0.13,0.13}{##1}}}
\expandafter\def\csname PY@tok@si\endcsname{\let\PY@bf=\textbf\def\PY@tc##1{\textcolor[rgb]{0.73,0.40,0.53}{##1}}}
\expandafter\def\csname PY@tok@se\endcsname{\let\PY@bf=\textbf\def\PY@tc##1{\textcolor[rgb]{0.73,0.40,0.13}{##1}}}
\expandafter\def\csname PY@tok@sr\endcsname{\def\PY@tc##1{\textcolor[rgb]{0.73,0.40,0.53}{##1}}}
\expandafter\def\csname PY@tok@ss\endcsname{\def\PY@tc##1{\textcolor[rgb]{0.10,0.09,0.49}{##1}}}
\expandafter\def\csname PY@tok@sx\endcsname{\def\PY@tc##1{\textcolor[rgb]{0.00,0.50,0.00}{##1}}}
\expandafter\def\csname PY@tok@m\endcsname{\def\PY@tc##1{\textcolor[rgb]{0.40,0.40,0.40}{##1}}}
\expandafter\def\csname PY@tok@gh\endcsname{\let\PY@bf=\textbf\def\PY@tc##1{\textcolor[rgb]{0.00,0.00,0.50}{##1}}}
\expandafter\def\csname PY@tok@gu\endcsname{\let\PY@bf=\textbf\def\PY@tc##1{\textcolor[rgb]{0.50,0.00,0.50}{##1}}}
\expandafter\def\csname PY@tok@gd\endcsname{\def\PY@tc##1{\textcolor[rgb]{0.63,0.00,0.00}{##1}}}
\expandafter\def\csname PY@tok@gi\endcsname{\def\PY@tc##1{\textcolor[rgb]{0.00,0.63,0.00}{##1}}}
\expandafter\def\csname PY@tok@gr\endcsname{\def\PY@tc##1{\textcolor[rgb]{1.00,0.00,0.00}{##1}}}
\expandafter\def\csname PY@tok@ge\endcsname{\let\PY@it=\textit}
\expandafter\def\csname PY@tok@gs\endcsname{\let\PY@bf=\textbf}
\expandafter\def\csname PY@tok@gp\endcsname{\let\PY@bf=\textbf\def\PY@tc##1{\textcolor[rgb]{0.00,0.00,0.50}{##1}}}
\expandafter\def\csname PY@tok@go\endcsname{\def\PY@tc##1{\textcolor[rgb]{0.53,0.53,0.53}{##1}}}
\expandafter\def\csname PY@tok@gt\endcsname{\def\PY@tc##1{\textcolor[rgb]{0.00,0.27,0.87}{##1}}}
\expandafter\def\csname PY@tok@err\endcsname{\def\PY@bc##1{\setlength{\fboxsep}{0pt}\fcolorbox[rgb]{1.00,0.00,0.00}{1,1,1}{\strut ##1}}}
\expandafter\def\csname PY@tok@kc\endcsname{\let\PY@bf=\textbf\def\PY@tc##1{\textcolor[rgb]{0.00,0.50,0.00}{##1}}}
\expandafter\def\csname PY@tok@kd\endcsname{\let\PY@bf=\textbf\def\PY@tc##1{\textcolor[rgb]{0.00,0.50,0.00}{##1}}}
\expandafter\def\csname PY@tok@kn\endcsname{\let\PY@bf=\textbf\def\PY@tc##1{\textcolor[rgb]{0.00,0.50,0.00}{##1}}}
\expandafter\def\csname PY@tok@kr\endcsname{\let\PY@bf=\textbf\def\PY@tc##1{\textcolor[rgb]{0.00,0.50,0.00}{##1}}}
\expandafter\def\csname PY@tok@bp\endcsname{\def\PY@tc##1{\textcolor[rgb]{0.00,0.50,0.00}{##1}}}
\expandafter\def\csname PY@tok@fm\endcsname{\def\PY@tc##1{\textcolor[rgb]{0.00,0.00,1.00}{##1}}}
\expandafter\def\csname PY@tok@vc\endcsname{\def\PY@tc##1{\textcolor[rgb]{0.10,0.09,0.49}{##1}}}
\expandafter\def\csname PY@tok@vg\endcsname{\def\PY@tc##1{\textcolor[rgb]{0.10,0.09,0.49}{##1}}}
\expandafter\def\csname PY@tok@vi\endcsname{\def\PY@tc##1{\textcolor[rgb]{0.10,0.09,0.49}{##1}}}
\expandafter\def\csname PY@tok@vm\endcsname{\def\PY@tc##1{\textcolor[rgb]{0.10,0.09,0.49}{##1}}}
\expandafter\def\csname PY@tok@sa\endcsname{\def\PY@tc##1{\textcolor[rgb]{0.73,0.13,0.13}{##1}}}
\expandafter\def\csname PY@tok@sb\endcsname{\def\PY@tc##1{\textcolor[rgb]{0.73,0.13,0.13}{##1}}}
\expandafter\def\csname PY@tok@sc\endcsname{\def\PY@tc##1{\textcolor[rgb]{0.73,0.13,0.13}{##1}}}
\expandafter\def\csname PY@tok@dl\endcsname{\def\PY@tc##1{\textcolor[rgb]{0.73,0.13,0.13}{##1}}}
\expandafter\def\csname PY@tok@s2\endcsname{\def\PY@tc##1{\textcolor[rgb]{0.73,0.13,0.13}{##1}}}
\expandafter\def\csname PY@tok@sh\endcsname{\def\PY@tc##1{\textcolor[rgb]{0.73,0.13,0.13}{##1}}}
\expandafter\def\csname PY@tok@s1\endcsname{\def\PY@tc##1{\textcolor[rgb]{0.73,0.13,0.13}{##1}}}
\expandafter\def\csname PY@tok@mb\endcsname{\def\PY@tc##1{\textcolor[rgb]{0.40,0.40,0.40}{##1}}}
\expandafter\def\csname PY@tok@mf\endcsname{\def\PY@tc##1{\textcolor[rgb]{0.40,0.40,0.40}{##1}}}
\expandafter\def\csname PY@tok@mh\endcsname{\def\PY@tc##1{\textcolor[rgb]{0.40,0.40,0.40}{##1}}}
\expandafter\def\csname PY@tok@mi\endcsname{\def\PY@tc##1{\textcolor[rgb]{0.40,0.40,0.40}{##1}}}
\expandafter\def\csname PY@tok@il\endcsname{\def\PY@tc##1{\textcolor[rgb]{0.40,0.40,0.40}{##1}}}
\expandafter\def\csname PY@tok@mo\endcsname{\def\PY@tc##1{\textcolor[rgb]{0.40,0.40,0.40}{##1}}}
\expandafter\def\csname PY@tok@ch\endcsname{\let\PY@it=\textit\def\PY@tc##1{\textcolor[rgb]{0.25,0.50,0.50}{##1}}}
\expandafter\def\csname PY@tok@cm\endcsname{\let\PY@it=\textit\def\PY@tc##1{\textcolor[rgb]{0.25,0.50,0.50}{##1}}}
\expandafter\def\csname PY@tok@cpf\endcsname{\let\PY@it=\textit\def\PY@tc##1{\textcolor[rgb]{0.25,0.50,0.50}{##1}}}
\expandafter\def\csname PY@tok@c1\endcsname{\let\PY@it=\textit\def\PY@tc##1{\textcolor[rgb]{0.25,0.50,0.50}{##1}}}
\expandafter\def\csname PY@tok@cs\endcsname{\let\PY@it=\textit\def\PY@tc##1{\textcolor[rgb]{0.25,0.50,0.50}{##1}}}

\def\PYZbs{\char`\\}
\def\PYZus{\char`\_}
\def\PYZob{\char`\{}
\def\PYZcb{\char`\}}
\def\PYZca{\char`\^}
\def\PYZam{\char`\&}
\def\PYZlt{\char`\<}
\def\PYZgt{\char`\>}
\def\PYZsh{\char`\#}
\def\PYZpc{\char`\%}
\def\PYZdl{\char`\$}
\def\PYZhy{\char`\-}
\def\PYZsq{\char`\'}
\def\PYZdq{\char`\"}
\def\PYZti{\char`\~}
% for compatibility with earlier versions
\def\PYZat{@}
\def\PYZlb{[}
\def\PYZrb{]}
\makeatother


    % Exact colors from NB
    \definecolor{incolor}{rgb}{0.0, 0.0, 0.5}
    \definecolor{outcolor}{rgb}{0.545, 0.0, 0.0}



    
    % Prevent overflowing lines due to hard-to-break entities
    \sloppy 
    % Setup hyperref package
    \hypersetup{
      breaklinks=true,  % so long urls are correctly broken across lines
      colorlinks=true,
      urlcolor=urlcolor,
      linkcolor=linkcolor,
      citecolor=citecolor,
      }
    % Slightly bigger margins than the latex defaults
    
    \geometry{verbose,tmargin=1in,bmargin=1in,lmargin=1in,rmargin=1in}
    
    

    \begin{document}
    
    
    \maketitle
    
    

    
    For this project we'll be using pandas plotting functionality along with
Jupyter notebook interface to explore data quickly through
visualizations.

The goal is to demonstrate data visualization techniques by analyzing
job success outcomes of students who graduated from college between 2010
and 2012. The original data of on the job outcomes was released by
\href{https://www.census.gov/programs-surveys/acs/}{American Community
Survey}, which conducts surveys and aggregates the data. The website
FiveThirtyEight cleaned the dataset and released it on their
\href{https://github.com/fivethirtyeight/data/tree/master/college-majors}{GitHub
repo}.

Each row in the dataset represents a different major in college and
contains information on gender diversity, employment rates, median,
salaries, and more. Below lists some information provided in the data
dictionary:

    \begin{itemize}
\tightlist
\item
  \texttt{Rank} - Rank by median earnings (the dataset is ordered by
  this column)
\item
  \texttt{Major\_code} - Major code
\item
  \texttt{Major} - Major description
\item
  \texttt{Major\_categor} - Category of major
\item
  \texttt{Total} - Total number of people with major
\item
  \texttt{Sample\_size} - Sample size (unweighted) of full-time
\item
  \texttt{Men} - Male graduates
\item
  \texttt{Women} - Female graduates
\item
  \texttt{ShareWomen} - Women as share of total
\item
  \texttt{Employed} - Number employed
\item
  \texttt{Median} - Median salary of full-time, year-round workers.
\item
  \texttt{Low\_wage\_jobs} - Number in low-wage service jobs
\item
  \texttt{Full\_time} - Number employed 35 hours or more
\item
  \texttt{Part\_time} - Number employed less than 35 hours
\end{itemize}

    Using visualizations, we will explore the following questions from the
dataset:

    \begin{itemize}
\tightlist
\item
  Do students in more popular majors make more money?

  \begin{itemize}
  \tightlist
  \item
    Using scatter plots
  \end{itemize}
\item
  How many majors are predominantly male? Predominantly female?

  \begin{itemize}
  \tightlist
  \item
    Using histograms
  \end{itemize}
\item
  Which category of majors have the most students?

  \begin{itemize}
  \tightlist
  \item
    Using bar plots
  \end{itemize}
\end{itemize}

    \begin{Verbatim}[commandchars=\\\{\}]
{\color{incolor}In [{\color{incolor}1}]:} \PY{k+kn}{import} \PY{n+nn}{pandas} \PY{k}{as} \PY{n+nn}{pd}
        \PY{k+kn}{import} \PY{n+nn}{matplotlib}\PY{n+nn}{.}\PY{n+nn}{pyplot} \PY{k}{as} \PY{n+nn}{plt}
        \PY{o}{\PYZpc{}} \PY{n}{matplotlib} \PY{n}{inline}
\end{Verbatim}


    \begin{Verbatim}[commandchars=\\\{\}]
{\color{incolor}In [{\color{incolor}2}]:} \PY{n}{recent\PYZus{}grads} \PY{o}{=} \PY{n}{pd}\PY{o}{.}\PY{n}{read\PYZus{}csv}\PY{p}{(}\PY{l+s+s1}{\PYZsq{}}\PY{l+s+s1}{recent\PYZhy{}grads.csv}\PY{l+s+s1}{\PYZsq{}}\PY{p}{)}
\end{Verbatim}


    \section{Initial Dataset Exploration}\label{initial-dataset-exploration}

    \begin{Verbatim}[commandchars=\\\{\}]
{\color{incolor}In [{\color{incolor}3}]:} \PY{n}{display}\PY{p}{(}\PY{n}{recent\PYZus{}grads}\PY{o}{.}\PY{n}{iloc}\PY{p}{[}\PY{l+m+mi}{0}\PY{p}{]}\PY{p}{)}
\end{Verbatim}


    
    \begin{verbatim}
Rank                                        1
Major_code                               2419
Major                   PETROLEUM ENGINEERING
Total                                    2339
Men                                      2057
Women                                     282
Major_category                    Engineering
ShareWomen                           0.120564
Sample_size                                36
Employed                                 1976
Full_time                                1849
Part_time                                 270
Full_time_year_round                     1207
Unemployed                                 37
Unemployment_rate                   0.0183805
Median                                 110000
P25th                                   95000
P75th                                  125000
College_jobs                             1534
Non_college_jobs                          364
Low_wage_jobs                             193
Name: 0, dtype: object
    \end{verbatim}

    
    \begin{Verbatim}[commandchars=\\\{\}]
{\color{incolor}In [{\color{incolor}4}]:} \PY{n}{display}\PY{p}{(}\PY{n}{recent\PYZus{}grads}\PY{o}{.}\PY{n}{head}\PY{p}{(}\PY{p}{)}\PY{p}{)}
\end{Verbatim}


    
    \begin{verbatim}
   Rank  Major_code                                      Major    Total  \
0     1        2419                      PETROLEUM ENGINEERING   2339.0   
1     2        2416             MINING AND MINERAL ENGINEERING    756.0   
2     3        2415                  METALLURGICAL ENGINEERING    856.0   
3     4        2417  NAVAL ARCHITECTURE AND MARINE ENGINEERING   1258.0   
4     5        2405                       CHEMICAL ENGINEERING  32260.0   

       Men    Women Major_category  ShareWomen  Sample_size  Employed  \
0   2057.0    282.0    Engineering    0.120564           36      1976   
1    679.0     77.0    Engineering    0.101852            7       640   
2    725.0    131.0    Engineering    0.153037            3       648   
3   1123.0    135.0    Engineering    0.107313           16       758   
4  21239.0  11021.0    Engineering    0.341631          289     25694   

       ...        Part_time  Full_time_year_round  Unemployed  \
0      ...              270                  1207          37   
1      ...              170                   388          85   
2      ...              133                   340          16   
3      ...              150                   692          40   
4      ...             5180                 16697        1672   

   Unemployment_rate  Median  P25th   P75th  College_jobs  Non_college_jobs  \
0           0.018381  110000  95000  125000          1534               364   
1           0.117241   75000  55000   90000           350               257   
2           0.024096   73000  50000  105000           456               176   
3           0.050125   70000  43000   80000           529               102   
4           0.061098   65000  50000   75000         18314              4440   

   Low_wage_jobs  
0            193  
1             50  
2              0  
3              0  
4            972  

[5 rows x 21 columns]
    \end{verbatim}

    
    \begin{Verbatim}[commandchars=\\\{\}]
{\color{incolor}In [{\color{incolor}5}]:} \PY{n}{display}\PY{p}{(}\PY{n}{recent\PYZus{}grads}\PY{o}{.}\PY{n}{tail}\PY{p}{(}\PY{p}{)}\PY{p}{)}
\end{Verbatim}


    
    \begin{verbatim}
     Rank  Major_code                   Major   Total     Men   Women  \
168   169        3609                 ZOOLOGY  8409.0  3050.0  5359.0   
169   170        5201  EDUCATIONAL PSYCHOLOGY  2854.0   522.0  2332.0   
170   171        5202     CLINICAL PSYCHOLOGY  2838.0   568.0  2270.0   
171   172        5203   COUNSELING PSYCHOLOGY  4626.0   931.0  3695.0   
172   173        3501         LIBRARY SCIENCE  1098.0   134.0   964.0   

               Major_category  ShareWomen  Sample_size  Employed  \
168    Biology & Life Science    0.637293           47      6259   
169  Psychology & Social Work    0.817099            7      2125   
170  Psychology & Social Work    0.799859           13      2101   
171  Psychology & Social Work    0.798746           21      3777   
172                 Education    0.877960            2       742   

         ...        Part_time  Full_time_year_round  Unemployed  \
168      ...             2190                  3602         304   
169      ...              572                  1211         148   
170      ...              648                  1293         368   
171      ...              965                  2738         214   
172      ...              237                   410          87   

     Unemployment_rate  Median  P25th  P75th  College_jobs  Non_college_jobs  \
168           0.046320   26000  20000  39000          2771              2947   
169           0.065112   25000  24000  34000          1488               615   
170           0.149048   25000  25000  40000           986               870   
171           0.053621   23400  19200  26000          2403              1245   
172           0.104946   22000  20000  22000           288               338   

     Low_wage_jobs  
168            743  
169             82  
170            622  
171            308  
172            192  

[5 rows x 21 columns]
    \end{verbatim}

    
    \begin{Verbatim}[commandchars=\\\{\}]
{\color{incolor}In [{\color{incolor}6}]:} \PY{n}{recent\PYZus{}grads}\PY{o}{.}\PY{n}{describe}\PY{p}{(}\PY{p}{)}
\end{Verbatim}


\begin{Verbatim}[commandchars=\\\{\}]
{\color{outcolor}Out[{\color{outcolor}6}]:}              Rank   Major\_code          Total            Men          Women  \textbackslash{}
        count  173.000000   173.000000     172.000000     172.000000     172.000000   
        mean    87.000000  3879.815029   39370.081395   16723.406977   22646.674419   
        std     50.084928  1687.753140   63483.491009   28122.433474   41057.330740   
        min      1.000000  1100.000000     124.000000     119.000000       0.000000   
        25\%     44.000000  2403.000000    4549.750000    2177.500000    1778.250000   
        50\%     87.000000  3608.000000   15104.000000    5434.000000    8386.500000   
        75\%    130.000000  5503.000000   38909.750000   14631.000000   22553.750000   
        max    173.000000  6403.000000  393735.000000  173809.000000  307087.000000   
        
               ShareWomen  Sample\_size       Employed      Full\_time      Part\_time  \textbackslash{}
        count  172.000000   173.000000     173.000000     173.000000     173.000000   
        mean     0.522223   356.080925   31192.763006   26029.306358    8832.398844   
        std      0.231205   618.361022   50675.002241   42869.655092   14648.179473   
        min      0.000000     2.000000       0.000000     111.000000       0.000000   
        25\%      0.336026    39.000000    3608.000000    3154.000000    1030.000000   
        50\%      0.534024   130.000000   11797.000000   10048.000000    3299.000000   
        75\%      0.703299   338.000000   31433.000000   25147.000000    9948.000000   
        max      0.968954  4212.000000  307933.000000  251540.000000  115172.000000   
        
               Full\_time\_year\_round    Unemployed  Unemployment\_rate         Median  \textbackslash{}
        count            173.000000    173.000000         173.000000     173.000000   
        mean           19694.427746   2416.329480           0.068191   40151.445087   
        std            33160.941514   4112.803148           0.030331   11470.181802   
        min              111.000000      0.000000           0.000000   22000.000000   
        25\%             2453.000000    304.000000           0.050306   33000.000000   
        50\%             7413.000000    893.000000           0.067961   36000.000000   
        75\%            16891.000000   2393.000000           0.087557   45000.000000   
        max           199897.000000  28169.000000           0.177226  110000.000000   
        
                      P25th          P75th   College\_jobs  Non\_college\_jobs  \textbackslash{}
        count    173.000000     173.000000     173.000000        173.000000   
        mean   29501.445087   51494.219653   12322.635838      13284.497110   
        std     9166.005235   14906.279740   21299.868863      23789.655363   
        min    18500.000000   22000.000000       0.000000          0.000000   
        25\%    24000.000000   42000.000000    1675.000000       1591.000000   
        50\%    27000.000000   47000.000000    4390.000000       4595.000000   
        75\%    33000.000000   60000.000000   14444.000000      11783.000000   
        max    95000.000000  125000.000000  151643.000000     148395.000000   
        
               Low\_wage\_jobs  
        count     173.000000  
        mean     3859.017341  
        std      6944.998579  
        min         0.000000  
        25\%       340.000000  
        50\%      1231.000000  
        75\%      3466.000000  
        max     48207.000000  
\end{Verbatim}
            
    \subsubsection{Dropping Missing Values}\label{dropping-missing-values}

    \begin{Verbatim}[commandchars=\\\{\}]
{\color{incolor}In [{\color{incolor}7}]:} \PY{n}{raw\PYZus{}data\PYZus{}count} \PY{o}{=} \PY{n+nb}{len}\PY{p}{(}\PY{n}{recent\PYZus{}grads}\PY{p}{)}
        \PY{n+nb}{print}\PY{p}{(}\PY{n}{raw\PYZus{}data\PYZus{}count}\PY{p}{)}
\end{Verbatim}


    \begin{Verbatim}[commandchars=\\\{\}]
173

    \end{Verbatim}

    \begin{Verbatim}[commandchars=\\\{\}]
{\color{incolor}In [{\color{incolor}8}]:} \PY{n}{recent\PYZus{}grads} \PY{o}{=} \PY{n}{recent\PYZus{}grads}\PY{o}{.}\PY{n}{dropna}\PY{p}{(}\PY{p}{)}
        \PY{n}{cleaned\PYZus{}data\PYZus{}count} \PY{o}{=} \PY{n+nb}{len}\PY{p}{(}\PY{n}{recent\PYZus{}grads}\PY{p}{)}
        \PY{n+nb}{print}\PY{p}{(}\PY{n}{cleaned\PYZus{}data\PYZus{}count}\PY{p}{)}
\end{Verbatim}


    \begin{Verbatim}[commandchars=\\\{\}]
172

    \end{Verbatim}

    \begin{itemize}
\tightlist
\item
  Only one row contained missing values
\end{itemize}

    \section{Exploring Differences Between Majors: Scatter
Plots}\label{exploring-differences-between-majors-scatter-plots}

    Questions: * Do students in more popular majors make more money? * Do
students that majored in subjects that were majority female make more
money? * Is there a link between the number of full-time employees and
median salary?

    \subsubsection{Popular Majors}\label{popular-majors}

    \begin{Verbatim}[commandchars=\\\{\}]
{\color{incolor}In [{\color{incolor}9}]:} \PY{n}{recent\PYZus{}grads}\PY{o}{.}\PY{n}{plot}\PY{p}{(}\PY{n}{x}\PY{o}{=}\PY{l+s+s1}{\PYZsq{}}\PY{l+s+s1}{Total}\PY{l+s+s1}{\PYZsq{}}\PY{p}{,} \PY{n}{y}\PY{o}{=}\PY{l+s+s1}{\PYZsq{}}\PY{l+s+s1}{Median}\PY{l+s+s1}{\PYZsq{}}\PY{p}{,} \PY{n}{kind}\PY{o}{=}\PY{l+s+s1}{\PYZsq{}}\PY{l+s+s1}{scatter}\PY{l+s+s1}{\PYZsq{}}\PY{p}{,} \PY{n}{title} \PY{o}{=} \PY{l+s+s2}{\PYZdq{}}\PY{l+s+s2}{Median(Salary) vs Total(\PYZsh{} of graduates)}\PY{l+s+s2}{\PYZdq{}}\PY{p}{)}
\end{Verbatim}


\begin{Verbatim}[commandchars=\\\{\}]
{\color{outcolor}Out[{\color{outcolor}9}]:} <matplotlib.axes.\_subplots.AxesSubplot at 0x1181d4470>
\end{Verbatim}
            
    \begin{center}
    \adjustimage{max size={0.9\linewidth}{0.9\paperheight}}{output_18_1.png}
    \end{center}
    { \hspace*{\fill} \\}
    
    The above scatter plot shows that there is not a strong correlation
between popularity of a major and median salary of full time workers. A
surprising data point actually shows that the most popular major has a
median salary of full time workers that is in the low end around 36,000.
We should keep in mind that this dataset was collected from 2010 to 2012
and salaries reflected may be slightly higher if the dataset were
collected today. Lets take a closer look at the top five majors and the
median salaries associated with them.

    \begin{Verbatim}[commandchars=\\\{\}]
{\color{incolor}In [{\color{incolor}10}]:} \PY{n}{top\PYZus{}5\PYZus{}majors}\PY{o}{=} \PY{n}{recent\PYZus{}grads}\PY{o}{.}\PY{n}{sort\PYZus{}values}\PY{p}{(}\PY{p}{[}\PY{l+s+s1}{\PYZsq{}}\PY{l+s+s1}{Total}\PY{l+s+s1}{\PYZsq{}}\PY{p}{]}\PY{p}{,} \PY{n}{ascending} \PY{o}{=} \PY{k+kc}{False}\PY{p}{)}\PY{o}{.}\PY{n}{head}\PY{p}{(}\PY{p}{)}
\end{Verbatim}


    \begin{Verbatim}[commandchars=\\\{\}]
{\color{incolor}In [{\color{incolor}11}]:} \PY{n}{top\PYZus{}5\PYZus{}majors}
\end{Verbatim}


\begin{Verbatim}[commandchars=\\\{\}]
{\color{outcolor}Out[{\color{outcolor}11}]:}      Rank  Major\_code                                   Major     Total  \textbackslash{}
         145   146        5200                              PSYCHOLOGY  393735.0   
         76     77        6203  BUSINESS MANAGEMENT AND ADMINISTRATION  329927.0   
         123   124        3600                                 BIOLOGY  280709.0   
         57     58        6200                        GENERAL BUSINESS  234590.0   
         93     94        1901                          COMMUNICATIONS  213996.0   
         
                   Men     Women               Major\_category  ShareWomen  Sample\_size  \textbackslash{}
         145   86648.0  307087.0     Psychology \& Social Work    0.779933         2584   
         76   173809.0  156118.0                     Business    0.473190         4212   
         123  111762.0  168947.0       Biology \& Life Science    0.601858         1370   
         57   132238.0  102352.0                     Business    0.436302         2380   
         93    70619.0  143377.0  Communications \& Journalism    0.669999         2394   
         
              Employed      {\ldots}        Part\_time  Full\_time\_year\_round  Unemployed  \textbackslash{}
         145    307933      {\ldots}           115172                174438       28169   
         76     276234      {\ldots}            50357                199897       21502   
         123    182295      {\ldots}            72371                100336       13874   
         57     190183      {\ldots}            36241                138299       14946   
         93     179633      {\ldots}            49889                116251       14602   
         
              Unemployment\_rate  Median  P25th  P75th  College\_jobs  Non\_college\_jobs  \textbackslash{}
         145           0.083811   31500  24000  41000        125148            141860   
         76            0.072218   38000  29000  50000         36720            148395   
         123           0.070725   33400  24000  45000         88232             81109   
         57            0.072861   40000  30000  55000         29334            100831   
         93            0.075177   35000  27000  45000         40763             97964   
         
              Low\_wage\_jobs  
         145          48207  
         76           32395  
         123          28339  
         57           27320  
         93           27440  
         
         [5 rows x 21 columns]
\end{Verbatim}
            
    We can see from the above DataFrame the top five majors and the salaries
associated with full time workers are: * \texttt{Psychology} : 31,500 *
\texttt{Business\ Admin}: 38,000 * \texttt{Biology}: 33,400 *
\texttt{Business}: 40,000 * \texttt{Communications}: 35,000

    The average salary of the top 5 majors is \texttt{35,590}.

    \begin{Verbatim}[commandchars=\\\{\}]
{\color{incolor}In [{\color{incolor}12}]:} \PY{n}{avg\PYZus{}salary\PYZus{}top\PYZus{}5} \PY{o}{=} \PY{n}{top\PYZus{}5\PYZus{}majors}\PY{p}{[}\PY{l+s+s1}{\PYZsq{}}\PY{l+s+s1}{Median}\PY{l+s+s1}{\PYZsq{}}\PY{p}{]}\PY{o}{.}\PY{n}{mean}\PY{p}{(}\PY{p}{)}
         \PY{n+nb}{print}\PY{p}{(}\PY{n}{avg\PYZus{}salary\PYZus{}top\PYZus{}5}\PY{p}{)}
\end{Verbatim}


    \begin{Verbatim}[commandchars=\\\{\}]
35580.0

    \end{Verbatim}

    We'll compare this with the bottom 5 least popular majors below:

    \begin{Verbatim}[commandchars=\\\{\}]
{\color{incolor}In [{\color{incolor}13}]:} \PY{n}{bottom\PYZus{}5\PYZus{}majors} \PY{o}{=} \PY{n}{recent\PYZus{}grads}\PY{o}{.}\PY{n}{sort\PYZus{}values}\PY{p}{(}\PY{p}{[}\PY{l+s+s1}{\PYZsq{}}\PY{l+s+s1}{Total}\PY{l+s+s1}{\PYZsq{}}\PY{p}{]}\PY{p}{)}\PY{o}{.}\PY{n}{head}\PY{p}{(}\PY{p}{)}
         \PY{n}{display}\PY{p}{(}\PY{n}{bottom\PYZus{}5\PYZus{}majors}\PY{p}{)}
\end{Verbatim}


    
    \begin{verbatim}
     Rank  Major_code                                   Major  Total    Men  \
73     74        3801                   MILITARY TECHNOLOGIES  124.0  124.0   
52     53        4005        MATHEMATICS AND COMPUTER SCIENCE  609.0  500.0   
112   113        1106                            SOIL SCIENCE  685.0  476.0   
33     34        2411  GEOLOGICAL AND GEOPHYSICAL ENGINEERING  720.0  488.0   
1       2        2416          MINING AND MINERAL ENGINEERING  756.0  679.0   

     Women                       Major_category  ShareWomen  Sample_size  \
73     0.0  Industrial Arts & Consumer Services    0.000000            4   
52   109.0              Computers & Mathematics    0.178982            7   
112  209.0      Agriculture & Natural Resources    0.305109            4   
33   232.0                          Engineering    0.322222            5   
1     77.0                          Engineering    0.101852            7   

     Employed      ...        Part_time  Full_time_year_round  Unemployed  \
73          0      ...                0                   111           0   
52        559      ...                0                   391           0   
112       613      ...              185                   383           0   
33        604      ...              126                   396          49   
1         640      ...              170                   388          85   

     Unemployment_rate  Median  P25th  P75th  College_jobs  Non_college_jobs  \
73            0.000000   40000  40000  40000             0                 0   
52            0.000000   42000  30000  78000           452                67   
112           0.000000   35000  18500  44000           355               144   
33            0.075038   50000  42800  57000           501                50   
1             0.117241   75000  55000  90000           350               257   

     Low_wage_jobs  
73               0  
52              25  
112              0  
33              49  
1               50  

[5 rows x 21 columns]
    \end{verbatim}

    
    \begin{Verbatim}[commandchars=\\\{\}]
{\color{incolor}In [{\color{incolor}14}]:} \PY{n}{recent\PYZus{}grads}\PY{o}{.}\PY{n}{plot}\PY{p}{(}\PY{n}{x}\PY{o}{=}\PY{l+s+s1}{\PYZsq{}}\PY{l+s+s1}{Sample\PYZus{}size}\PY{l+s+s1}{\PYZsq{}}\PY{p}{,} \PY{n}{y}\PY{o}{=}\PY{l+s+s1}{\PYZsq{}}\PY{l+s+s1}{Unemployment\PYZus{}rate}\PY{l+s+s1}{\PYZsq{}}\PY{p}{,} \PY{n}{kind}\PY{o}{=}\PY{l+s+s1}{\PYZsq{}}\PY{l+s+s1}{scatter}\PY{l+s+s1}{\PYZsq{}}\PY{p}{,} \PY{n}{title} \PY{o}{=} \PY{l+s+s2}{\PYZdq{}}\PY{l+s+s2}{Unemployment Rate vs Sample Size}\PY{l+s+s2}{\PYZdq{}}\PY{p}{)}
\end{Verbatim}


\begin{Verbatim}[commandchars=\\\{\}]
{\color{outcolor}Out[{\color{outcolor}14}]:} <matplotlib.axes.\_subplots.AxesSubplot at 0x1184ecda0>
\end{Verbatim}
            
    \begin{center}
    \adjustimage{max size={0.9\linewidth}{0.9\paperheight}}{output_27_1.png}
    \end{center}
    { \hspace*{\fill} \\}
    
    The 5 least popular college majors and the salaries associated with them
are: * \texttt{Military\ Technologies}: 40,000 *
\texttt{Mathematics\ and\ Computer\ Science}: 78,000 *
\texttt{Soil\ Science}: 35000 *
\texttt{Geological\ and\ Geophysical\ Engineering}: 50,000 *
\texttt{Mining\ and\ Mineral\ Engineering}: 90,000

    \begin{Verbatim}[commandchars=\\\{\}]
{\color{incolor}In [{\color{incolor}15}]:} \PY{n}{avg\PYZus{}salary\PYZus{}bottom\PYZus{}5} \PY{o}{=} \PY{n}{bottom\PYZus{}5\PYZus{}majors}\PY{p}{[}\PY{l+s+s1}{\PYZsq{}}\PY{l+s+s1}{Median}\PY{l+s+s1}{\PYZsq{}}\PY{p}{]}\PY{o}{.}\PY{n}{mean}\PY{p}{(}\PY{p}{)}
         \PY{n+nb}{print}\PY{p}{(}\PY{n}{avg\PYZus{}salary\PYZus{}bottom\PYZus{}5}\PY{p}{)}
\end{Verbatim}


    \begin{Verbatim}[commandchars=\\\{\}]
48400.0

    \end{Verbatim}

    \subsubsection{Bar Charts: Popular/Unpopular
Majors}\label{bar-charts-popularunpopular-majors}

    \begin{Verbatim}[commandchars=\\\{\}]
{\color{incolor}In [{\color{incolor}16}]:} \PY{n}{top\PYZus{}5\PYZus{}majors}\PY{o}{.}\PY{n}{plot}\PY{o}{.}\PY{n}{bar}\PY{p}{(}\PY{n}{x}\PY{o}{=}\PY{l+s+s1}{\PYZsq{}}\PY{l+s+s1}{Major}\PY{l+s+s1}{\PYZsq{}}\PY{p}{,} \PY{n}{y}\PY{o}{=}\PY{l+s+s1}{\PYZsq{}}\PY{l+s+s1}{ShareWomen}\PY{l+s+s1}{\PYZsq{}}\PY{p}{,} \PY{n}{legend}\PY{o}{=}\PY{k+kc}{False}\PY{p}{)}
         \PY{n}{bottom\PYZus{}5\PYZus{}majors}\PY{o}{.}\PY{n}{plot}\PY{o}{.}\PY{n}{bar}\PY{p}{(}\PY{n}{x}\PY{o}{=}\PY{l+s+s1}{\PYZsq{}}\PY{l+s+s1}{Major}\PY{l+s+s1}{\PYZsq{}}\PY{p}{,} \PY{n}{y}\PY{o}{=}\PY{l+s+s1}{\PYZsq{}}\PY{l+s+s1}{ShareWomen}\PY{l+s+s1}{\PYZsq{}}\PY{p}{,} \PY{n}{legend}\PY{o}{=}\PY{k+kc}{False}\PY{p}{)}
\end{Verbatim}


\begin{Verbatim}[commandchars=\\\{\}]
{\color{outcolor}Out[{\color{outcolor}16}]:} <matplotlib.axes.\_subplots.AxesSubplot at 0x118a31278>
\end{Verbatim}
            
    \begin{center}
    \adjustimage{max size={0.9\linewidth}{0.9\paperheight}}{output_31_1.png}
    \end{center}
    { \hspace*{\fill} \\}
    
    \begin{center}
    \adjustimage{max size={0.9\linewidth}{0.9\paperheight}}{output_31_2.png}
    \end{center}
    { \hspace*{\fill} \\}
    
    \begin{Verbatim}[commandchars=\\\{\}]
{\color{incolor}In [{\color{incolor}17}]:} \PY{n}{top\PYZus{}5\PYZus{}majors}\PY{o}{.}\PY{n}{plot}\PY{o}{.}\PY{n}{bar}\PY{p}{(}\PY{n}{x}\PY{o}{=}\PY{l+s+s1}{\PYZsq{}}\PY{l+s+s1}{Major}\PY{l+s+s1}{\PYZsq{}}\PY{p}{,} \PY{n}{y}\PY{o}{=}\PY{l+s+s1}{\PYZsq{}}\PY{l+s+s1}{Median}\PY{l+s+s1}{\PYZsq{}}\PY{p}{,} \PY{n}{legend}\PY{o}{=}\PY{k+kc}{False}\PY{p}{)}
         \PY{n}{bottom\PYZus{}5\PYZus{}majors}\PY{o}{.}\PY{n}{plot}\PY{o}{.}\PY{n}{bar}\PY{p}{(}\PY{n}{x}\PY{o}{=}\PY{l+s+s1}{\PYZsq{}}\PY{l+s+s1}{Major}\PY{l+s+s1}{\PYZsq{}}\PY{p}{,} \PY{n}{y}\PY{o}{=}\PY{l+s+s1}{\PYZsq{}}\PY{l+s+s1}{Median}\PY{l+s+s1}{\PYZsq{}}\PY{p}{,} \PY{n}{legend}\PY{o}{=}\PY{k+kc}{False}\PY{p}{)}
\end{Verbatim}


\begin{Verbatim}[commandchars=\\\{\}]
{\color{outcolor}Out[{\color{outcolor}17}]:} <matplotlib.axes.\_subplots.AxesSubplot at 0x1189fac50>
\end{Verbatim}
            
    \begin{center}
    \adjustimage{max size={0.9\linewidth}{0.9\paperheight}}{output_32_1.png}
    \end{center}
    { \hspace*{\fill} \\}
    
    \begin{center}
    \adjustimage{max size={0.9\linewidth}{0.9\paperheight}}{output_32_2.png}
    \end{center}
    { \hspace*{\fill} \\}
    
    The average salary of the 5 least popular majors is \texttt{48,400},
about 13,000 more than the top 5 most popular majors. This could mean
that there is more competition in the employment marketplace for these
employees as a result of fewer graduates. These majors could also
include skills that are difficult to master making them valuable as well
as resulting in less graduates.

    \subsubsection{Note Majors With Highest Salaries(Just for
fun)}\label{note-majors-with-highest-salariesjust-for-fun}

    \begin{Verbatim}[commandchars=\\\{\}]
{\color{incolor}In [{\color{incolor}33}]:} \PY{n}{Highest\PYZus{}salary} \PY{o}{=} \PY{n}{recent\PYZus{}grads}\PY{o}{.}\PY{n}{sort\PYZus{}values}\PY{p}{(}\PY{p}{[}\PY{l+s+s1}{\PYZsq{}}\PY{l+s+s1}{Median}\PY{l+s+s1}{\PYZsq{}}\PY{p}{]}\PY{p}{,} \PY{n}{ascending} \PY{o}{=} \PY{k+kc}{False}\PY{p}{)}
\end{Verbatim}


    \begin{Verbatim}[commandchars=\\\{\}]
{\color{incolor}In [{\color{incolor}35}]:} \PY{n}{Highest\PYZus{}salary}\PY{o}{.}\PY{n}{head}\PY{p}{(}\PY{l+m+mi}{20}\PY{p}{)}
\end{Verbatim}


\begin{Verbatim}[commandchars=\\\{\}]
{\color{outcolor}Out[{\color{outcolor}35}]:}     Rank  Major\_code                                      Major    Total  \textbackslash{}
         0      1        2419                      PETROLEUM ENGINEERING   2339.0   
         1      2        2416             MINING AND MINERAL ENGINEERING    756.0   
         2      3        2415                  METALLURGICAL ENGINEERING    856.0   
         3      4        2417  NAVAL ARCHITECTURE AND MARINE ENGINEERING   1258.0   
         4      5        2405                       CHEMICAL ENGINEERING  32260.0   
         5      6        2418                        NUCLEAR ENGINEERING   2573.0   
         6      7        6202                          ACTUARIAL SCIENCE   3777.0   
         7      8        5001                 ASTRONOMY AND ASTROPHYSICS   1792.0   
         10    11        2407                       COMPUTER ENGINEERING  41542.0   
         13    14        5008                          MATERIALS SCIENCE   4279.0   
         12    13        2404                     BIOMEDICAL ENGINEERING  14955.0   
         9     10        2408                     ELECTRICAL ENGINEERING  81527.0   
         8      9        2414                     MECHANICAL ENGINEERING  91227.0   
         11    12        2401                      AEROSPACE ENGINEERING  15058.0   
         14    15        2409  ENGINEERING MECHANICS PHYSICS AND SCIENCE   4321.0   
         15    16        2402                     BIOLOGICAL ENGINEERING   8925.0   
         16    17        2412   INDUSTRIAL AND MANUFACTURING ENGINEERING  18968.0   
         17    18        2400                        GENERAL ENGINEERING  61152.0   
         18    19        2403                  ARCHITECTURAL ENGINEERING   2825.0   
         19    20        3201                            COURT REPORTING   1148.0   
         
                 Men    Women       Major\_category  ShareWomen  Sample\_size  Employed  \textbackslash{}
         0    2057.0    282.0          Engineering    0.120564           36      1976   
         1     679.0     77.0          Engineering    0.101852            7       640   
         2     725.0    131.0          Engineering    0.153037            3       648   
         3    1123.0    135.0          Engineering    0.107313           16       758   
         4   21239.0  11021.0          Engineering    0.341631          289     25694   
         5    2200.0    373.0          Engineering    0.144967           17      1857   
         6    2110.0   1667.0             Business    0.441356           51      2912   
         7     832.0    960.0    Physical Sciences    0.535714           10      1526   
         10  33258.0   8284.0          Engineering    0.199413          399     32506   
         13   2949.0   1330.0          Engineering    0.310820           22      3307   
         12   8407.0   6548.0          Engineering    0.437847           79     10047   
         9   65511.0  16016.0          Engineering    0.196450          631     61928   
         8   80320.0  10907.0          Engineering    0.119559         1029     76442   
         11  12953.0   2105.0          Engineering    0.139793          147     11391   
         14   3526.0    795.0          Engineering    0.183985           30      3608   
         15   6062.0   2863.0          Engineering    0.320784           55      6170   
         16  12453.0   6515.0          Engineering    0.343473          183     15604   
         17  45683.0  15469.0          Engineering    0.252960          425     44931   
         18   1835.0    990.0          Engineering    0.350442           26      2575   
         19    877.0    271.0  Law \& Public Policy    0.236063           14       930   
         
                 {\ldots}        Part\_time  Full\_time\_year\_round  Unemployed  \textbackslash{}
         0       {\ldots}              270                  1207          37   
         1       {\ldots}              170                   388          85   
         2       {\ldots}              133                   340          16   
         3       {\ldots}              150                   692          40   
         4       {\ldots}             5180                 16697        1672   
         5       {\ldots}              264                  1449         400   
         6       {\ldots}              296                  2482         308   
         7       {\ldots}              553                   827          33   
         10      {\ldots}             5146                 23621        2275   
         13      {\ldots}              878                  1967          78   
         12      {\ldots}             2694                  5986        1019   
         9       {\ldots}            12695                 41413        3895   
         8       {\ldots}            13101                 54639        4650   
         11      {\ldots}             2724                  8790         794   
         14      {\ldots}              811                  2004          23   
         15      {\ldots}             1983                  3413         589   
         16      {\ldots}             2243                 11326         699   
         17      {\ldots}             7199                 33540        2859   
         18      {\ldots}              343                  1848         170   
         19      {\ldots}              223                   808          11   
         
             Unemployment\_rate  Median  P25th   P75th  College\_jobs  Non\_college\_jobs  \textbackslash{}
         0            0.018381  110000  95000  125000          1534               364   
         1            0.117241   75000  55000   90000           350               257   
         2            0.024096   73000  50000  105000           456               176   
         3            0.050125   70000  43000   80000           529               102   
         4            0.061098   65000  50000   75000         18314              4440   
         5            0.177226   65000  50000  102000          1142               657   
         6            0.095652   62000  53000   72000          1768               314   
         7            0.021167   62000  31500  109000           972               500   
         10           0.065409   60000  45000   75000         23694              5721   
         13           0.023043   60000  39000   65000          2626               391   
         12           0.092084   60000  36000   70000          6439              2471   
         9            0.059174   60000  45000   72000         45829             10874   
         8            0.057342   60000  48000   70000         52844             16384   
         11           0.065162   60000  42000   70000          8184              2425   
         14           0.006334   58000  25000   74000          2439               947   
         15           0.087143   57100  40000   76000          3603              1595   
         16           0.042876   57000  37900   67000          8306              3235   
         17           0.059824   56000  36000   69000         26898             11734   
         18           0.061931   54000  38000   65000          1665               649   
         19           0.011690   54000  50000   54000           402               528   
         
             Low\_wage\_jobs  
         0             193  
         1              50  
         2               0  
         3               0  
         4             972  
         5             244  
         6             259  
         7             220  
         10            980  
         13             81  
         12            789  
         9            3170  
         8            3253  
         11            372  
         14            263  
         15            524  
         16            640  
         17           3192  
         18            137  
         19            144  
         
         [20 rows x 21 columns]
\end{Verbatim}
            
    \begin{Verbatim}[commandchars=\\\{\}]
{\color{incolor}In [{\color{incolor}18}]:} \PY{n}{avg\PYZus{}salary\PYZus{}bottom\PYZus{}5} \PY{o}{=} \PY{n}{bottom\PYZus{}5\PYZus{}majors}\PY{p}{[}\PY{l+s+s1}{\PYZsq{}}\PY{l+s+s1}{Median}\PY{l+s+s1}{\PYZsq{}}\PY{p}{]}\PY{o}{.}\PY{n}{mean}\PY{p}{(}\PY{p}{)}
         \PY{n+nb}{print}\PY{p}{(}\PY{n}{avg\PYZus{}salary\PYZus{}bottom\PYZus{}5}\PY{p}{)}
\end{Verbatim}


    \begin{Verbatim}[commandchars=\\\{\}]
48400.0

    \end{Verbatim}

    \subsubsection{Majority Female Majors}\label{majority-female-majors}

    \begin{Verbatim}[commandchars=\\\{\}]
{\color{incolor}In [{\color{incolor}19}]:} \PY{n}{recent\PYZus{}grads}\PY{o}{.}\PY{n}{plot}\PY{p}{(}\PY{n}{x}\PY{o}{=}\PY{l+s+s1}{\PYZsq{}}\PY{l+s+s1}{ShareWomen}\PY{l+s+s1}{\PYZsq{}}\PY{p}{,} \PY{n}{y}\PY{o}{=}\PY{l+s+s1}{\PYZsq{}}\PY{l+s+s1}{Unemployment\PYZus{}rate}\PY{l+s+s1}{\PYZsq{}}\PY{p}{,} \PY{n}{kind}\PY{o}{=}\PY{l+s+s1}{\PYZsq{}}\PY{l+s+s1}{scatter}\PY{l+s+s1}{\PYZsq{}}\PY{p}{,} \PY{n}{title}\PY{o}{=}\PY{l+s+s1}{\PYZsq{}}\PY{l+s+s1}{Share of Women vs Unemployment Rate}\PY{l+s+s1}{\PYZsq{}}\PY{p}{)}
\end{Verbatim}


\begin{Verbatim}[commandchars=\\\{\}]
{\color{outcolor}Out[{\color{outcolor}19}]:} <matplotlib.axes.\_subplots.AxesSubplot at 0x118842b38>
\end{Verbatim}
            
    \begin{center}
    \adjustimage{max size={0.9\linewidth}{0.9\paperheight}}{output_39_1.png}
    \end{center}
    { \hspace*{\fill} \\}
    
    \begin{itemize}
\tightlist
\item
  No correlation between share of woman and unemployment rate
\end{itemize}

    \begin{Verbatim}[commandchars=\\\{\}]
{\color{incolor}In [{\color{incolor}20}]:} \PY{n}{recent\PYZus{}grads}\PY{o}{.}\PY{n}{plot}\PY{p}{(}\PY{n}{x}\PY{o}{=}\PY{l+s+s1}{\PYZsq{}}\PY{l+s+s1}{Men}\PY{l+s+s1}{\PYZsq{}}\PY{p}{,} \PY{n}{y}\PY{o}{=}\PY{l+s+s1}{\PYZsq{}}\PY{l+s+s1}{Median}\PY{l+s+s1}{\PYZsq{}}\PY{p}{,} \PY{n}{kind}\PY{o}{=}\PY{l+s+s1}{\PYZsq{}}\PY{l+s+s1}{scatter}\PY{l+s+s1}{\PYZsq{}}\PY{p}{,} \PY{n}{title}\PY{o}{=}\PY{l+s+s1}{\PYZsq{}}\PY{l+s+s1}{Men(\PYZsh{} of male graduates) vs Median(Avg Salary )}\PY{l+s+s1}{\PYZsq{}}\PY{p}{)}
\end{Verbatim}


\begin{Verbatim}[commandchars=\\\{\}]
{\color{outcolor}Out[{\color{outcolor}20}]:} <matplotlib.axes.\_subplots.AxesSubplot at 0x118ba56d8>
\end{Verbatim}
            
    \begin{center}
    \adjustimage{max size={0.9\linewidth}{0.9\paperheight}}{output_41_1.png}
    \end{center}
    { \hspace*{\fill} \\}
    
    \begin{Verbatim}[commandchars=\\\{\}]
{\color{incolor}In [{\color{incolor}21}]:} \PY{n}{recent\PYZus{}grads}\PY{o}{.}\PY{n}{plot}\PY{p}{(}\PY{n}{x}\PY{o}{=}\PY{l+s+s1}{\PYZsq{}}\PY{l+s+s1}{Women}\PY{l+s+s1}{\PYZsq{}}\PY{p}{,} \PY{n}{y}\PY{o}{=}\PY{l+s+s1}{\PYZsq{}}\PY{l+s+s1}{Median}\PY{l+s+s1}{\PYZsq{}}\PY{p}{,} \PY{n}{kind}\PY{o}{=}\PY{l+s+s1}{\PYZsq{}}\PY{l+s+s1}{scatter}\PY{l+s+s1}{\PYZsq{}}\PY{p}{,} \PY{n}{title}\PY{o}{=}\PY{l+s+s1}{\PYZsq{}}\PY{l+s+s1}{Women(\PYZsh{} of female graduates) vs Median(Avg Salary)}\PY{l+s+s1}{\PYZsq{}}\PY{p}{)}
\end{Verbatim}


\begin{Verbatim}[commandchars=\\\{\}]
{\color{outcolor}Out[{\color{outcolor}21}]:} <matplotlib.axes.\_subplots.AxesSubplot at 0x1185707f0>
\end{Verbatim}
            
    \begin{center}
    \adjustimage{max size={0.9\linewidth}{0.9\paperheight}}{output_42_1.png}
    \end{center}
    { \hspace*{\fill} \\}
    
    \begin{Verbatim}[commandchars=\\\{\}]
{\color{incolor}In [{\color{incolor}22}]:} \PY{n}{recent\PYZus{}grads}\PY{o}{.}\PY{n}{plot}\PY{p}{(}\PY{n}{x}\PY{o}{=}\PY{l+s+s1}{\PYZsq{}}\PY{l+s+s1}{ShareWomen}\PY{l+s+s1}{\PYZsq{}}\PY{p}{,} \PY{n}{y}\PY{o}{=}\PY{l+s+s1}{\PYZsq{}}\PY{l+s+s1}{Median}\PY{l+s+s1}{\PYZsq{}}\PY{p}{,} \PY{n}{kind} \PY{o}{=} \PY{l+s+s1}{\PYZsq{}}\PY{l+s+s1}{scatter}\PY{l+s+s1}{\PYZsq{}}\PY{p}{,} \PY{n}{title} \PY{o}{=}\PY{l+s+s1}{\PYZsq{}}\PY{l+s+s1}{Share of Women vs Average Salary}\PY{l+s+s1}{\PYZsq{}}\PY{p}{)}
\end{Verbatim}


\begin{Verbatim}[commandchars=\\\{\}]
{\color{outcolor}Out[{\color{outcolor}22}]:} <matplotlib.axes.\_subplots.AxesSubplot at 0x118820198>
\end{Verbatim}
            
    \begin{center}
    \adjustimage{max size={0.9\linewidth}{0.9\paperheight}}{output_43_1.png}
    \end{center}
    { \hspace*{\fill} \\}
    
    We can see from the above chart that as the number of women in a college
major increases the average salary of full time workers seems to
decrease. This could mean the women are choosing majors that aren't as
sought after by employers. Lets look at the top 10 majors with the
biggest share of women:

    \begin{Verbatim}[commandchars=\\\{\}]
{\color{incolor}In [{\color{incolor}23}]:} \PY{n}{top10\PYZus{}women} \PY{o}{=} \PY{n}{recent\PYZus{}grads}\PY{o}{.}\PY{n}{sort\PYZus{}values}\PY{p}{(}\PY{p}{[}\PY{l+s+s1}{\PYZsq{}}\PY{l+s+s1}{ShareWomen}\PY{l+s+s1}{\PYZsq{}}\PY{p}{]}\PY{p}{,} \PY{n}{ascending} \PY{o}{=} \PY{k+kc}{False}\PY{p}{)}\PY{o}{.}\PY{n}{head}\PY{p}{(}\PY{l+m+mi}{10}\PY{p}{)}
         \PY{n}{display}\PY{p}{(}\PY{n}{top10\PYZus{}women}\PY{p}{)}
\end{Verbatim}


    
    \begin{verbatim}
     Rank  Major_code                                          Major  \
164   165        2307                      EARLY CHILDHOOD EDUCATION   
163   164        6102  COMMUNICATION DISORDERS SCIENCES AND SERVICES   
51     52        6104                     MEDICAL ASSISTING SERVICES   
138   139        2304                           ELEMENTARY EDUCATION   
150   151        2901                   FAMILY AND CONSUMER SCIENCES   
100   101        2310                        SPECIAL NEEDS EDUCATION   
156   157        5403      HUMAN SERVICES AND COMMUNITY ORGANIZATION   
151   152        5404                                    SOCIAL WORK   
34     35        6107                                        NURSING   
88     89        6199       MISCELLANEOUS HEALTH MEDICAL PROFESSIONS   

        Total      Men     Women                       Major_category  \
164   37589.0   1167.0   36422.0                            Education   
163   38279.0   1225.0   37054.0                               Health   
51    11123.0    803.0   10320.0                               Health   
138  170862.0  13029.0  157833.0                            Education   
150   58001.0   5166.0   52835.0  Industrial Arts & Consumer Services   
100   28739.0   2682.0   26057.0                            Education   
156    9374.0    885.0    8489.0             Psychology & Social Work   
151   53552.0   5137.0   48415.0             Psychology & Social Work   
34   209394.0  21773.0  187621.0                               Health   
88    13386.0   1589.0   11797.0                               Health   

     ShareWomen  Sample_size  Employed      ...        Part_time  \
164    0.968954          342     32551      ...             7001   
163    0.967998           95     29763      ...            13862   
51     0.927807           67      9168      ...             4107   
138    0.923745         1629    149339      ...            37965   
150    0.910933          518     46624      ...            15872   
100    0.906677          246     24639      ...             5153   
156    0.905590           89      8294      ...             2405   
151    0.904075          374     45038      ...            13481   
34     0.896019         2554    180903      ...            40818   
88     0.881294           81     10076      ...             4145   

     Full_time_year_round  Unemployed  Unemployment_rate  Median  P25th  \
164                 20748        1360           0.040105   28000  21000   
163                 14460        1487           0.047584   28000  20000   
51                   4290         407           0.042507   42000  30000   
138                 86540        7297           0.046586   32000  23400   
150                 26906        3355           0.067128   30000  22900   
100                 16642        1067           0.041508   35000  32000   
156                  5061         326           0.037819   30000  24000   
151                 27588        3329           0.068828   30000  25000   
34                 122817        8497           0.044863   48000  39000   
88                   5868         893           0.081411   36000  23000   

     P75th  College_jobs  Non_college_jobs  Low_wage_jobs  
164  35000         23515              7705           2868  
163  40000         19957              9404           5125  
51   65000          2091              6948           1270  
138  38000        108085             36972          11502  
150  40000         20985             20133           5248  
100  42000         20185              3797           1179  
156  35000          2878              4595            724  
151  35000         27449             14416           4344  
34   58000        151643             26146           6193  
88   42000          5652              3835           1422  

[10 rows x 21 columns]
    \end{verbatim}

    
    \subsubsection{Full Time Employees vs Median
Salaries}\label{full-time-employees-vs-median-salaries}

    \begin{Verbatim}[commandchars=\\\{\}]
{\color{incolor}In [{\color{incolor}24}]:} \PY{n}{recent\PYZus{}grads}\PY{o}{.}\PY{n}{plot}\PY{p}{(}\PY{n}{x}\PY{o}{=}\PY{l+s+s1}{\PYZsq{}}\PY{l+s+s1}{Full\PYZus{}time}\PY{l+s+s1}{\PYZsq{}}\PY{p}{,} \PY{n}{y}\PY{o}{=}\PY{l+s+s1}{\PYZsq{}}\PY{l+s+s1}{Median}\PY{l+s+s1}{\PYZsq{}}\PY{p}{,} \PY{n}{kind}\PY{o}{=}\PY{l+s+s1}{\PYZsq{}}\PY{l+s+s1}{Scatter}\PY{l+s+s1}{\PYZsq{}}\PY{p}{,}\PY{n}{title}\PY{o}{=}\PY{l+s+s1}{\PYZsq{}}\PY{l+s+s1}{\PYZsh{} Full Time Employees vs Average Salary}\PY{l+s+s1}{\PYZsq{}}\PY{p}{)}
\end{Verbatim}


\begin{Verbatim}[commandchars=\\\{\}]
{\color{outcolor}Out[{\color{outcolor}24}]:} <matplotlib.axes.\_subplots.AxesSubplot at 0x118ed8b00>
\end{Verbatim}
            
    \begin{center}
    \adjustimage{max size={0.9\linewidth}{0.9\paperheight}}{output_47_1.png}
    \end{center}
    { \hspace*{\fill} \\}
    
    There is some correlation between the number of full time employees and
average salaries. As average salary increases the number of full time
employees decreases for some majors. There could be number of reason for
this. Employers may only need a smaller number of highly skilled workers
that are paid more.

    \section{Exploring Column Distributions:
Histograms}\label{exploring-column-distributions-histograms}

    \begin{itemize}
\tightlist
\item
  What is the most common median salary range?
\item
  What percent of majors are predominantly male? Predominantly female?
\end{itemize}

    \subsubsection{Common Median Salary
Range}\label{common-median-salary-range}

    \begin{Verbatim}[commandchars=\\\{\}]
{\color{incolor}In [{\color{incolor}25}]:} \PY{n}{recent\PYZus{}grads}\PY{p}{[}\PY{l+s+s1}{\PYZsq{}}\PY{l+s+s1}{Median}\PY{l+s+s1}{\PYZsq{}}\PY{p}{]}\PY{o}{.}\PY{n}{hist}\PY{p}{(}\PY{n}{bins}\PY{o}{=}\PY{l+m+mi}{15}\PY{p}{,} \PY{n+nb}{range}\PY{o}{=}\PY{p}{(}\PY{l+m+mi}{0}\PY{p}{,}\PY{l+m+mi}{100000}\PY{p}{)}\PY{p}{)}
\end{Verbatim}


\begin{Verbatim}[commandchars=\\\{\}]
{\color{outcolor}Out[{\color{outcolor}25}]:} <matplotlib.axes.\_subplots.AxesSubplot at 0x11902a2b0>
\end{Verbatim}
            
    \begin{center}
    \adjustimage{max size={0.9\linewidth}{0.9\paperheight}}{output_52_1.png}
    \end{center}
    { \hspace*{\fill} \\}
    
    The above histogram shows that the the most common median salary for
recent graduates is in the 35,000 to 40,000 range.

    \subsubsection{Male}\label{male}

    \begin{Verbatim}[commandchars=\\\{\}]
{\color{incolor}In [{\color{incolor}26}]:} \PY{n}{recent\PYZus{}grads}\PY{p}{[}\PY{l+s+s1}{\PYZsq{}}\PY{l+s+s1}{Men}\PY{l+s+s1}{\PYZsq{}}\PY{p}{]}\PY{o}{.}\PY{n}{hist}\PY{p}{(}\PY{n}{bins} \PY{o}{=} \PY{l+m+mi}{20}\PY{p}{,} \PY{n+nb}{range}\PY{o}{=}\PY{p}{(}\PY{l+m+mi}{0}\PY{p}{,}\PY{l+m+mi}{150000}\PY{p}{)}\PY{p}{)}
\end{Verbatim}


\begin{Verbatim}[commandchars=\\\{\}]
{\color{outcolor}Out[{\color{outcolor}26}]:} <matplotlib.axes.\_subplots.AxesSubplot at 0x1191155c0>
\end{Verbatim}
            
    \begin{center}
    \adjustimage{max size={0.9\linewidth}{0.9\paperheight}}{output_55_1.png}
    \end{center}
    { \hspace*{\fill} \\}
    
    \subsubsection{Female}\label{female}

    \begin{Verbatim}[commandchars=\\\{\}]
{\color{incolor}In [{\color{incolor}27}]:} \PY{n}{recent\PYZus{}grads}\PY{p}{[}\PY{l+s+s1}{\PYZsq{}}\PY{l+s+s1}{Women}\PY{l+s+s1}{\PYZsq{}}\PY{p}{]}\PY{o}{.}\PY{n}{hist}\PY{p}{(}\PY{n}{bins} \PY{o}{=} \PY{l+m+mi}{20}\PY{p}{,} \PY{n+nb}{range}\PY{o}{=}\PY{p}{(}\PY{l+m+mi}{0}\PY{p}{,}\PY{l+m+mi}{200000}\PY{p}{)}\PY{p}{)}
\end{Verbatim}


\begin{Verbatim}[commandchars=\\\{\}]
{\color{outcolor}Out[{\color{outcolor}27}]:} <matplotlib.axes.\_subplots.AxesSubplot at 0x118c9df28>
\end{Verbatim}
            
    \begin{center}
    \adjustimage{max size={0.9\linewidth}{0.9\paperheight}}{output_57_1.png}
    \end{center}
    { \hspace*{\fill} \\}
    
    \subsubsection{Share Female}\label{share-female}

    \begin{Verbatim}[commandchars=\\\{\}]
{\color{incolor}In [{\color{incolor}28}]:} \PY{n}{recent\PYZus{}grads}\PY{p}{[}\PY{l+s+s1}{\PYZsq{}}\PY{l+s+s1}{ShareWomen}\PY{l+s+s1}{\PYZsq{}}\PY{p}{]}\PY{o}{.}\PY{n}{hist}\PY{p}{(}\PY{n}{bins} \PY{o}{=} \PY{l+m+mi}{10}\PY{p}{,} \PY{n+nb}{range}\PY{o}{=} \PY{p}{(}\PY{l+m+mi}{0}\PY{p}{,}\PY{l+m+mi}{1}\PY{p}{)}\PY{p}{)}
\end{Verbatim}


\begin{Verbatim}[commandchars=\\\{\}]
{\color{outcolor}Out[{\color{outcolor}28}]:} <matplotlib.axes.\_subplots.AxesSubplot at 0x119125ef0>
\end{Verbatim}
            
    \begin{center}
    \adjustimage{max size={0.9\linewidth}{0.9\paperheight}}{output_59_1.png}
    \end{center}
    { \hspace*{\fill} \\}
    
    The above histogram shows that a large percentage of college majors have
a majority of female students graduating. We will compare plots to gain
some more insight.

    \section{Scatter Matrix Plot
Comparison}\label{scatter-matrix-plot-comparison}

    \begin{Verbatim}[commandchars=\\\{\}]
{\color{incolor}In [{\color{incolor}29}]:} \PY{k+kn}{from} \PY{n+nn}{pandas}\PY{n+nn}{.}\PY{n+nn}{plotting} \PY{k}{import} \PY{n}{scatter\PYZus{}matrix}
\end{Verbatim}


    \begin{Verbatim}[commandchars=\\\{\}]
{\color{incolor}In [{\color{incolor}30}]:} \PY{n}{scatter\PYZus{}matrix}\PY{p}{(}\PY{n}{recent\PYZus{}grads}\PY{p}{[}\PY{p}{[}\PY{l+s+s1}{\PYZsq{}}\PY{l+s+s1}{Sample\PYZus{}size}\PY{l+s+s1}{\PYZsq{}}\PY{p}{,} \PY{l+s+s1}{\PYZsq{}}\PY{l+s+s1}{Median}\PY{l+s+s1}{\PYZsq{}}\PY{p}{]}\PY{p}{]}\PY{p}{,} \PY{n}{figsize}\PY{o}{=}\PY{p}{(}\PY{l+m+mi}{6}\PY{p}{,}\PY{l+m+mi}{6}\PY{p}{)}\PY{p}{)}
\end{Verbatim}


\begin{Verbatim}[commandchars=\\\{\}]
{\color{outcolor}Out[{\color{outcolor}30}]:} array([[<matplotlib.axes.\_subplots.AxesSubplot object at 0x1194a4518>,
                 <matplotlib.axes.\_subplots.AxesSubplot object at 0x119649470>],
                [<matplotlib.axes.\_subplots.AxesSubplot object at 0x1195b6b00>,
                 <matplotlib.axes.\_subplots.AxesSubplot object at 0x1196d01d0>]],
               dtype=object)
\end{Verbatim}
            
    \begin{center}
    \adjustimage{max size={0.9\linewidth}{0.9\paperheight}}{output_63_1.png}
    \end{center}
    { \hspace*{\fill} \\}
    
    \begin{Verbatim}[commandchars=\\\{\}]
{\color{incolor}In [{\color{incolor}31}]:} \PY{n}{scatter\PYZus{}matrix}\PY{p}{(}\PY{n}{recent\PYZus{}grads}\PY{p}{[}\PY{p}{[}\PY{l+s+s1}{\PYZsq{}}\PY{l+s+s1}{Sample\PYZus{}size}\PY{l+s+s1}{\PYZsq{}}\PY{p}{,} \PY{l+s+s1}{\PYZsq{}}\PY{l+s+s1}{Median}\PY{l+s+s1}{\PYZsq{}}\PY{p}{,} \PY{l+s+s1}{\PYZsq{}}\PY{l+s+s1}{Unemployment\PYZus{}rate}\PY{l+s+s1}{\PYZsq{}}\PY{p}{]}\PY{p}{]}\PY{p}{,} \PY{n}{figsize}\PY{o}{=}\PY{p}{(}\PY{l+m+mi}{10}\PY{p}{,}\PY{l+m+mi}{10}\PY{p}{)}\PY{p}{)}
\end{Verbatim}


\begin{Verbatim}[commandchars=\\\{\}]
{\color{outcolor}Out[{\color{outcolor}31}]:} array([[<matplotlib.axes.\_subplots.AxesSubplot object at 0x11979be80>,
                 <matplotlib.axes.\_subplots.AxesSubplot object at 0x11987c048>,
                 <matplotlib.axes.\_subplots.AxesSubplot object at 0x1198a46d8>],
                [<matplotlib.axes.\_subplots.AxesSubplot object at 0x1198cdd68>,
                 <matplotlib.axes.\_subplots.AxesSubplot object at 0x1198fe438>,
                 <matplotlib.axes.\_subplots.AxesSubplot object at 0x1198fe470>],
                [<matplotlib.axes.\_subplots.AxesSubplot object at 0x119958198>,
                 <matplotlib.axes.\_subplots.AxesSubplot object at 0x119980828>,
                 <matplotlib.axes.\_subplots.AxesSubplot object at 0x1199a8c18>]],
               dtype=object)
\end{Verbatim}
            
    \begin{center}
    \adjustimage{max size={0.9\linewidth}{0.9\paperheight}}{output_64_1.png}
    \end{center}
    { \hspace*{\fill} \\}
    
    \begin{Verbatim}[commandchars=\\\{\}]
{\color{incolor}In [{\color{incolor}32}]:} \PY{n}{scatter\PYZus{}matrix}\PY{p}{(}\PY{n}{recent\PYZus{}grads}\PY{p}{[}\PY{p}{[}\PY{l+s+s1}{\PYZsq{}}\PY{l+s+s1}{ShareWomen}\PY{l+s+s1}{\PYZsq{}}\PY{p}{,} \PY{l+s+s1}{\PYZsq{}}\PY{l+s+s1}{Median}\PY{l+s+s1}{\PYZsq{}}\PY{p}{]}\PY{p}{]}\PY{p}{,} \PY{n}{figsize}\PY{o}{=}\PY{p}{(}\PY{l+m+mi}{6}\PY{p}{,}\PY{l+m+mi}{6}\PY{p}{)}\PY{p}{)}
\end{Verbatim}


\begin{Verbatim}[commandchars=\\\{\}]
{\color{outcolor}Out[{\color{outcolor}32}]:} array([[<matplotlib.axes.\_subplots.AxesSubplot object at 0x119d3dbe0>,
                 <matplotlib.axes.\_subplots.AxesSubplot object at 0x119d62978>],
                [<matplotlib.axes.\_subplots.AxesSubplot object at 0x119f2b048>,
                 <matplotlib.axes.\_subplots.AxesSubplot object at 0x119f526d8>]],
               dtype=object)
\end{Verbatim}
            
    \begin{center}
    \adjustimage{max size={0.9\linewidth}{0.9\paperheight}}{output_65_1.png}
    \end{center}
    { \hspace*{\fill} \\}
    
    \begin{itemize}
\tightlist
\item
  Again we can see that as the percentage of woman in major decrease the
  median salary of full time workers increases.
\end{itemize}

    \section{Conclusions}\label{conclusions}

    The goal of this project was to demonstrate some of pandas data
visualization functionality through the analysis of college major data
collected by American Community Survey and cleaned by the popular
FiveThirtyEight website. We have demonstrated the use of scatter plots,
bar charts, histograms, and scatter matrix visualizations.

After analyzing the data we can clearly see the popular majors do not
have job outcomes that include higher salaries. In fact the most popular
major psychology has a low starting salary for graduates around 36,000.
Furthermore woman make woman make up a large percent of the most popular
majors with some negative correlation of share of women and median
salary outcomes.


    % Add a bibliography block to the postdoc
    
    
    
    \end{document}
